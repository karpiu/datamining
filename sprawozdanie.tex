
\documentclass[a4paper,12pt]{article}
\usepackage{amsfonts} % math symobls
\usepackage{amsthm}
\usepackage{graphicx} % importing EPS
\usepackage[cp1250]{inputenc}
\usepackage[OT4]{fontenc}
\usepackage[polish]{babel}
\linespread{1.0} % line spacing

\hyphenation{Eks-plo-ra-cji}

\newtheorem{defi}{Definicja}
\newtheorem{theo}[defi]{Twierdzenie}
\newtheorem{lemma}[defi]{Lemat}
\newtheorem{obs}[defi]{Obserwacja}

\title{Reguły asocjacyjne w analizie wybranych kursów walut}

\author{K.~Król\\
M.~K.~Karpiński\\
\\
Uniwersytet Wrocławski\\
Instytut Informatyki\\
SPRAWOZDANIE Z PROJEKTU}

\date{Wrocław, dnia \today\ r.}

\begin{document}
\thispagestyle{empty}
\maketitle

\newpage

\tableofcontents

\newpage


%\begin{figure}[h]
%\begin{center}$
%\begin{array}{cc}
%\includegraphics[scale=0.25,angle=0]{plot1.png} & \includegraphics[scale=0.25,angle=0]{plot2.png} \\
%\includegraphics[scale=0.25,angle=0]{plot3.png} & \includegraphics[scale=0.25,angle=0]{plot4.png}
%\end{array}$
%\caption{Wyresy funkcji, na których przeprowadzane s¹ testy.}
%\end{center}
%\end{figure}

\section{Wstęp}
\indent \indent Dokument ten zosta³ sporz¹dzony jako sprawozdanie do projektu z przedmiotu: Eksploracja Danych.

\section{Dane}

\section{Oprogramowanie}

\section{Algorytmy}

\section{Analiza Danych}

\section{Wnioski}

\end{document}

