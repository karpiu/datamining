
\documentclass[a4paper,12pt]{article}
\usepackage{graphicx}
\usepackage[polish]{babel}
\usepackage[utf8]{inputenc}
%\usepackage[cp1250]{inputenc} % to uzywamy na windowsie
\usepackage[T1]{fontenc}
%\usepackage[OT4]{fontenc} % to tez 
\usepackage{url}
\usepackage{hyperref}
\linespread{1.0}

\hyphenation{Eks-plo-ra-cji}

\title{Reguły asocjacyjne w analizie wybranych kursów walut}

\author{K.~Król\\
M.~K.~Karpiński\\
\\
Uniwersytet Wrocławski\\
Instytut Informatyki\\
SPRAWOZDANIE Z PROJEKTU}

\date{Wrocław, dnia \today\ r.}

\begin{document}
\thispagestyle{empty}
\maketitle

\newpage

\tableofcontents

\newpage


%\begin{figure}[h]
%\begin{center}$
%\begin{array}{cc}
%\includegraphics[scale=0.25,angle=0]{plot1.png} & \includegraphics[scale=0.25,angle=0]{plot2.png} \\
%\includegraphics[scale=0.25,angle=0]{plot3.png} & \includegraphics[scale=0.25,angle=0]{plot4.png}
%\end{array}$
%\caption{Wyresy funkcji, na których przeprowadzane są testy.}
%\end{center}
%\end{figure}

\section{Wstęp}
\indent \indent Dokument ten został sporządzony jako sprawozdanie do projektu z przedmiotu: Eksploracja Danych.

\section{Dane}

\indent \indent Dane wykorzystane w projeckie zostały pobrane ze strony \url{http://nlp.uned.es/social-tagging/wiki10+/}.
\section{Oprogramowanie}

\section{Algorytmy}

\section{Analiza Danych}

\section{Wnioski}

\end{document}

